% Problem ID: 060
% Original file: Training_1/contents/area.tex


\numbering 볼록 사각형 $ABCD$ 에서 대각선 $\overline{AC}, \overline{BD}$ 의 중점을 각각 $M,N$ 이라 하고, \\ 
$M,N$ 을 지나고 $\overline{BD}, \overline{AC}$ 에 평행한 두 직선의 교점을 $O$ 라 할 때, 
$O$ 를 각 변의 중점과 연결하면 사각형 $ABDC$ 의 넓이는 이들 네 선분에 의해 4등분 됨을 보여라.\\
\begin{center}
\begin{tikzpicture} 
	\draw[thick] (-2,-2) node[below left] {$A$} coordinate (A)
		-- (3, -2) node[below right] {$ B$} coordinate (B)
		-- (1, 2) node[above right] {$ C$} coordinate (C)
		-- (-1.25, 1) node[above left] {$ D$} coordinate (D) -- cycle  ;
		
	\draw[gray, thin] (A) -- (C)   (B) -- (D);
		
	\draw[blue, thin, fill] ({(-2+1)/2}, {(-2+2)/2}) circle (1pt) coordinate (M) node[left] {$ M$}
			( {(3-1.25)/2}, {(-2+1)/2}) circle (1pt) coordinate (N) node[above right] {$ N$};
			
	\path[name path=M--] (M) -- +($ (B)-(D) $);
	\path[name path=N--] (N) -- +($ (A)-(C) $);
	\path[name intersections={of=M-- and N--, by=O}];
	\draw (O) node[below left] {$ O$};
	
	\draw[densely dashed] (N) -- (O)  (O) -- (M);
	
	\draw[red, thick] (O) --($(A)!0.5!(B)$) node[below] {$ E$} coordinate (AB)
		(O) -- ($(B)!0.5!(C)$) node[right] {$ F$} coordinate (BC)
		(O) -- ($(C)!0.5!(D)$) node[above] {$ G$} coordinate (CD)
		(O) -- ($(D)!0.5!(A)$) node[left] {$ H$} coordinate (DA) -- cycle;
\end{tikzpicture}
\end{center}
