% Problem ID: 129
% Source: 24학년도 세종영재고 1번
% Original file: Training_1/1_Angle/0_inscribed_angle.tex

\numbering 중심이 $O$인 원에 내접하는 사각형 $ABCD$에서 $\angle AOB + \angle COD = 180^\circ$이다. 
다음 물음에 답하시오.\\
\begin{center}
    \begin{tikzpicture}
        \tkzDefPoint(0,0){O}
        \tkzDefCircle[R](O, 3)    \tkzGetPoint{X}
        \tkzDrawCircle(O,X)

        \tkzDefPoint(105:3){A}      \tkzDefPoint(135:3){B}
        \tkzDefPoint(190:3){C}      \tkzDefPoint(-10:3){D}
        \tkzDrawPolygon[thick](A, B, C, D)
        \tkzDrawSegments[thick](O,A O,B O,C O,D)

        \tkzLabelPoint[above](A){$A$}
        \tkzLabelPoint[left](C){$C$}
        \tkzLabelPoint[right](D){$D$}
    \end{tikzpicture}
\end{center}

(1) $\overline{AC}=2 \mybreak \overline{BD}=3$일 때, 사각형 $ABCD$의 넓이를 구하고 그 과정을 설명하시오.[4점]\\


(2) 두 선분 $AB, CD$의 길이가 모두 자연수이고 
$\overline{BC}^2 + \overline{DA}^2 = 65$일 때, 
$\overline{AB}, \overline{CD}$가 될 수 있는 값들을 모두 구하고 
그 과정을 설명하시오.[5점]\\