% Problem ID: 148
% Source: 28회 KMO 중등부 2차 1번
% Original file: Training_1/2_circle/1_concyclic.tex
% 1 phase
            %%%%%%%%%%%%%%%%%%%%%%%%%%%%%%%%%%%%%%%%%%%%%%%%%
            %%%%%%%%%%%%%%%%%%%%%%%%%%%%%%%%%%%%%%%%%%%%%%
            %%%%%%%%%%%%%%%%%%%%%%%%%%%%%%%%%%%%%%%%%%%%%%
            %%%%%%%%%%%%%%%%%%%%%%%%%%%%%%%%%%%%%%%%%%%%%%
            %%%%%%%%%%%%%%%%%%%%%%%%%%%%%%%%%%%%%%%%%%%%%%
            %\tkzDrawCircle[semithick, red, opacity=0.8](Z,W)
            %%%%%%%%%%%%%%%%%%%%%%%%%%%%%%%%%%%%%%%%%%%
            %\tkzDrawCircle[very thick, orange](tmp1,D)


\numbering 삼각형 $ABC$ 의 내심을 $I$ 라 하고, 
직선 $AI$ 가 변 $BC$ 와 만나는 점을 $D$ 라 하자.
삼각형 $ABD$ 의 내심 $E$ 와 $D$ 를 지나는 직선이 삼각형 $BCE$ 의 외접원과 만나는 점을 
$P(\not=E)$ 라 하자.
또, 삼각형 $ACD$ 의 내심 $F$ 와 $D$ 를 지나는 직선이 삼각형 $BCF$ 의 외접원과 만나는 점을 
$Q(\not=F)$ 라 하자.
변 $BC$ 의 중점이 삼각형 $DPQ$ 의 외접원 위에 있음을 보여라. \\
\tikzmath {   
    \R = 3; \myRatio = 0.3; \opac = 0.7;
    \b=-150; \c=-30; \a=70;    
    \aa= \a-180; \bb=\b+180; \cc=\c+180;
}

\begin{figure}[H]
	\begin{subfigure}{0.5\textwidth}
        \begin{center}
        \begin{tikzpicture}[scale=0.6]
            \tkzDefPoint(0,0){O}        %\tkzDrawPoint[label=right:$O$](O)

            \tkzDefPoint(45:\R){A}      \tkzDrawPoint[label=above:$ A$](A)   
            \tkzDefPoint(-160:\R){B}    \tkzDrawPoint[label=left:$ B$](B)
            \tkzDefPoint(-20:\R){C}     \tkzDrawPoint[label= right:$ C$](C)
            
            \tkzDrawPolygon[semithick](A,B,C)

            \tkzDefCircle[in](A,B,C)    \tkzGetPoint{I}
            \tkzInterLL(A,I)(B,C)       \tkzGetPoint{D}
            \tkzDrawPoint(D)
            \tkzLabelPoint[below right, xshift=0.1cm](D){$ D$}
            \tkzDrawSegment(A,D)

            \tkzDefCircle[in](A,B,D) \tkzGetPoint{E}
            \tkzDrawPoint[label=above left:$ E$](E)
            \tkzDefCircle[in](A,C,D) \tkzGetPoint{F}
            \tkzDrawPoint[label=above right:$ F$](F)

            \tkzLabelAngle[pos=0.5](E,D,B){$\times$}
            \tkzLabelAngle[pos=0.5](A,D,E){$\times$}
            \tkzLabelAngle[pos=0.5](F,D,A){$\circ$}
            \tkzLabelAngle[pos=0.5](C,D,F){$\circ$}
            
            \tkzDefCircle[circum](B,C,E) \tkzGetPoints{X}{Y}
            \tkzDrawCircle[thick, blue](X,Y)
            
            \tkzInterLC(E,D)(X,Y) \tkzGetFirstPoint{P}
            \tkzDrawPoint[blue, label=right:$ P$](P)
            \tkzDrawSegment(E,P)
            
            \tkzDefCircle[circum](B,C,F) \tkzGetPoints{Z}{W}
            \tkzDrawCircle[thick, red](Z,W)
            
            \tkzInterLC(F,D)(Z,W) \tkzGetSecondPoint{Q}
            \tkzDrawPoint[red, label=left:$ Q$](Q)
            \tkzDrawSegment(F,Q)
            
            \tkzMarkRightAngle(Q,D,P)
        \end{tikzpicture} 
        \subcaption{$\angle PDQ = 90^{\circ}$}
        \end{center}
    \end{subfigure}%
    \begin{subfigure}{0.5\textwidth}
        \begin{center}
        \begin{tikzpicture}[scale=0.6]
            \tkzDefPoint(0,0){O}        %\tkzDrawPoint[label=right:$ O$](O)
            \tkzDefPoint(45:\R){A}      \tkzDrawPoint[label=above:$ A$](A)   
            \tkzDefPoint(-160:\R){B}    \tkzDrawPoint[label=left:$ B$](B)
            \tkzDefPoint(-20:\R){C}     \tkzDrawPoint[label= right:$ C$](C)
            
            \tkzDrawCircle[semithick](O,A)
            \tkzDrawPolygon(A,B,C)

            \tkzDefCircle[in](A,B,C)    \tkzGetPoint{I}
            \tkzInterLL(A,I)(B,C)       \tkzGetPoint{D}
            \tkzDrawPoint(D)

            \tkzDefCircle[in](A,B,D) \tkzGetPoint{E}
            \tkzDrawPoint[label=above left:$ E$](E)
            \tkzDefCircle[in](A,C,D) \tkzGetPoint{F}
            \tkzDrawPoint[label=above right:$ F$](F)

            \tkzLabelAngle[pos=0.5](E,D,B){$\times$}
            \tkzLabelAngle[pos=0.5](A,D,E){$\times$}

            \tkzDefCircle[circum](B,C,E) \tkzGetPoints{X}{Y}
            \tkzDrawCircle[semithick, blue](X,Y)

            \tkzInterLC(E,D)(X,Y) \tkzGetFirstPoint{P}
            \tkzDrawPoint[label=right:$ P$](P)
            \tkzDrawSegment(E,P)

            \tkzDefCircle[circum](B,C,F) \tkzGetPoints{Z}{W}

            \tkzInterLC(F,D)(Z,W) \tkzGetSecondPoint{Q}
            \tkzDrawPoint[label=left:$ Q$](Q)
            \tkzDrawSegment(F,Q)

            \tkzDefCircle[ex](C,A,B) \tkzGetPoint{Ia}
            \tkzDrawPoint[label=below:$ I_a$](Ia)
            \tkzDrawPoint[label=above left:$ I$](I)
            \tkzDrawSegment(A,Ia)

            \tkzDefMidPoint(I,Ia)       \tkzGetPoint{M}
            \tkzDrawPoint[label=above left:$ M$](M)

            \tkzDrawSegments[very thick, red](B,I I,C)
            \tkzDrawSegments(C,Ia)
            \tkzMarkAngles[size=8mm](C,B,E C,P,D C,Ia,D)

            \tkzDrawSegments(C,P Ia,P) \tkzMarkSegments[mark=|||, size=2pt](C,P Ia,P)

            \tkzDefCircle[circum](D,C,P) \tkzGetPoint{tmp1}

            \tkzDefCircle[circum](D,B,Q) \tkzGetPoint{tmp2}

        \end{tikzpicture} 
        \subcaption{$\angle IBC = \angle EBC = \angle CPD$ 와 같은 것은?}
    \end{center}
    \end{subfigure}%

\end{figure}

왼쪽 그림은 설명에 등장하는 요소들을 최소한으로 그린 것이다.
문제를 해결하기 위해 찾아야 할 것은 무엇일까? \\

$ABD$, $ ACD$는 최초의 $ ABC$와 공유하는 요소를 가질 것이다.
오른쪽 그림에 $ ABC$와 $ ABD$, $ ABC$와 $ ACD$의 \fbox{공통 이등분선} 
$ IB$와 $ IC$를 표시하였다.
이로부터 파란색 원위 원주각을 $ ABC$의 방심과 
관련지을 수 있다. $\angle CPD = \angle II_aC = \angle DI_aC$ 이므로
$D,C,P,I_a$ 가 공원점임을 알게 된다.
