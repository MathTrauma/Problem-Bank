% Problem ID: 103
% Original file: PAS/분해_재구성.tex
	%\draw[help lines] (-3,-3) grid (5,4);
	%\draw[ultra thin] (B)--(D); 


\numbering 평행사변형 $ABCD$ 에서 점 $D$ 를 지나는 임의의 직선이
$\overline{BC}$ 와 $E, \overline{AB}$ 의 연장과 $F$ 에서 만나는 직선을 그으면
$\triangle ABE = \triangle CEF$ 가 성립함을 보여라.\\
\begin{center}
\resizebox{8cm}{8cm}{%
\begin{tikzpicture}
	\coordinate[label=above left:$A$] (A) at (1,3);
	\coordinate[label=below left:$ B$] (B) at (0,0);
	\coordinate[label=below right:$ C$] (C) at (4,0);
	\coordinate[label=above left:$ D$] (D) at (5,3);
	\draw[thick] (A) -- (B) -- (C) -- (D) -- cycle;

	\coordinate[label=below:$ E$] (E) at (1.5, 0);
	
	\coordinate (X) at ($(D)!2.5!(E)$);
	\path [name path=D--X] (D) -- (X);
    	\path [name path=A--] (A) -- ($(A)!2.0!(B)$);
	\path[name intersections={of=D--X and A--, by=F}];
	
	\draw[black, ultra thin] (D)--(F);
	\draw[ultra thin] (B)--(F);
	\node[label=left:$ F$] at (F) {};
	
	\draw[black, thick, fill=red!10, fill opacity=.2] (A) -- (B) -- (E) -- cycle; 
	\draw[black, thick, fill=red!10, fill opacity=.2] (C) -- (E) -- (F) -- cycle;

\end{tikzpicture} }
\end{center}
