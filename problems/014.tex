% Problem ID: 014
% Original file: Problem Sets/contents/20_좌표.tex
% \begin{figure}[H]
% 	\begin{subfigure}{0.49\textwidth}
%         \begin{center}
%             \begin{asy}
%                 size(6cm); import geometry;
%                 point A=(4,0), B=(0,2), C=(0,0);
%                 dot("$A$", A, S); dot("$B$", B, N); dot("$C$", C, W);
%                 draw(A--B--C--cycle);
%                 //drawPoint(A);
%                 point D=(4,0), E=(6,0), F=(6,2);
%                 draw(D--E--F--cycle);
%             \end{asy}
%         \end{center}
% 	\end{subfigure}  %%%%%%%%%
% 	\begin{subfigure}{0.49\textwidth}
%         \begin{center}
%             \begin{asy}
%                 size(6cm); import geometry;
%                 point A=(4,0), B=(0,2), C=(0,0);
%                 draw(A--B--C--cycle);
%                 //drawPoint(A);
%                 real t = 1;
%                 point D=(4-t,0), E=(6-t,0), F=(6-t,2);
%                 draw(D--E--F--cycle);
%             \end{asy}
%         \end{center}
% 	\end{subfigure}
	%%%%%%%%%%%%%%%%%%%%%%%%%%%%%%%%%
	%\caption{Caption}
%\end{figure}


\numbering 삼각형 $ABC$ 는 
$\overline{AC}=4, \overline{BC}=2 \mybreak \angle ACB=90^\circ$ 
를 만족하고 삼각형 $ DEF$는
$\overline{DE}=\overline{EF} = 2 \mybreak \angle DEF = 90^\circ$
를 만족한다.\\

다음 그림은 $ A,C$와 $ D,E$가 직선 $ l$ 위에 놓인 상태로 
두 삼각형이 미끄러지면서 움직이는 모습이다.\\
