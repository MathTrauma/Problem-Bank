% Problem ID: 156
% Source: 공통수학2 - 절대등급 22쪽 8번
% Original file: Training_1/3_technique/1_midpoint.tex
        %\tkzDrawCircle(o,P)
        %\tkzDrawArc


\numbering 그림과 같이 반지름의 길이가 8인 원을 접어
접힌 호가 지름 $AB$ 위의 점 $P$ 에서 접하도록 하였다.
원의 중심 $O$ 와 $P$ 사이의 거리가 6일 때, 
$O$ 와 접힌 호 위의 점 사이의 거리의 최솟값은? \\
\begin{center}
    \begin{tikzpicture}
        \useasboundingbox (-4.5, -4.5) rectangle (4.5,4.5);
        \tkzDefPoints{-4/0/A, 0/0/O, 4/0/B, 3/0/P, 3/4/o}
        \tkzDrawSegment(A,B)   
        \tkzInterCC(O,B)(o,P)   \tkzGetPoints{x}{y}
        \tkzDrawPoint(x)

        \tkzDrawArc[thick](O,x)(y)     \tkzDrawArc[dashed, thick](O,y)(x)
        \tkzDrawArc[semithick, red](o,x)(y)

        \tkzLabelPoint[left](A){$A$}
        \tkzLabelPoint[right](B){$ B$}
        \tkzLabelPoint[below left](O){$ O$}
        \tkzLabelPoint[below right](P){$ P$}
        \tkzDrawPoints(O,P)
        \tkzDrawSegment(x,y)
        \tkzDrawSegment[dim={$6$,-6pt,}](O,P)
    \end{tikzpicture} 
\end{center}
