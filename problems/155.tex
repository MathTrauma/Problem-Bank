% Problem ID: 155
% Original file: Training_1/3_technique/1_midpoint.tex
        %\useasboundingbox (-4.5, -4.5) rectangle (4.5,4.5);
        %%%%%%%%%%%%%%%%%%%%%%%%%%%%%%%%%%%%%%%%%


\numbering 그림과 같이 반지름의 길이가 1이고 중심각의 크기가 $90^\circ$ 인 부채꼴 $OAB$ 가 있다.
호 $AB$ 위의 점 $ C$에 대하여 선분 $ BC$를 지름으로 하는 원을 그린다.
선분 $ BC$의 중점을 지나고 직선 $ OB$에 평행한 직선이 원과 만나는 점 중 점 $ B$에 가까운 점을 $ P$라 하자. 
$\overline{BC}=x$ 일 때, 삼각형 $ OAP$의 넓이를 $ S(x)$라 하자. 
$ S(x)$ 의 최댓값이 $\frac q p$ 일 때, $p+q$ 의 값을 구하시오. 
(단, $0 \lt x \le \sqrt{2}$ 이고, $ p$와 $ q$는 서로소인 자연수이다.)\\

\begin{center}
    \begin{tikzpicture}[scale=1.2]
        \def\r{5}
        \tkzDefPoints{0/0/O, \r/0/A, 0/\r/B}
        \tkzDefPoint(0,6){B_1}
        \tkzDefPoint(40:\r){C}

        \tkzDefMidPoint(C,B)    \tkzGetPoint{Gamma}
        \tkzDefPointWith[colinear=at Gamma](O,B)    \tkzGetPoint{p}

        \tkzInterLC(Gamma,p)(Gamma,B)   \tkzGetPoints{P}{y}
        \tkzInterLL(Gamma,P)(O,A)   \tkzGetPoint{D}
        

        \tkzDrawPolygon[fill=gray!30, opacity=0.5](O,A,P)

        \tkzDrawPoints(O,A,B,C,P)
        \tkzLabelPoint[above right](C){$ C$}
        \tkzDrawArc(O,A)(B)

        \tkzDrawCircle(Gamma, B)
        \tkzDrawSegments(B,O O,A B,C D,P)

        \tkzLabelPoint[above](P){$ P$}
        \tkzLabelPoint[below left](O){$ O$}
        \tkzLabelPoint[below right](A){$ A$}
        \tkzLabelPoint[above left](B){$ B$}
    \end{tikzpicture} 
\end{center}
