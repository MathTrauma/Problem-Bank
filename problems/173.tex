% Problem ID: 173
% Original file: tkz_practice/tikz_과제.tex
%\begin{center}
%    \begin{tikzpicture}[scale=1.25, thick, help lines/.style={thin,draw=black!50}]
%        \tkzDefPoint(0,0){O}
%
%        \foreach \pos/\name in {130/A, -150/B, -30/C} {
%        \tkzDefPoint[label=\pos:$\name$](\pos:\R){\name}
%        }
%        \tkzDrawPoints(A, B, C)   \tkzDrawPolygon[thick](A,B,C)
%
%        \tkzDefMidPoint(B,C)  \tkzGetPoint{M}
%        \tkzDefLine[orthogonal=through M](B,C) \tkzGetPoint{D}
%
%        \tkzInterLC(M,D)(O,A) \tkzGetPoints{X}{Y}
%        \tkzDrawPoints(X,Y)
%        \tkzLabelPoints(X,Y)
%
%        \tkzDrawCircle(O,A)
%
%        \tkzDrawLine[dashed](A,X)
%    \end{tikzpicture}
%\end{center}
%


\numbering 세 점을 지나가는 원 그리기  : 함수로 만들 수는 없을까?
\begin{verbatim}
  \draw let \p1=($(X)!0.5!(A)$), \p2=($(X)!0.5!(B)$), 
      \p3=($(\p1)!1!90:(A)$), \p4=($(\p2)!1!-90:(B)$),	
      \p5=(intersection of \p1--\p3 and \p2--\p4)    
      in coordinate (O_1) at (\p5) ;
  \node[name path=C_1, circle through=(A), draw=blue] at (O_1) {};
\end{verbatim}
