% Problem ID: 107
% Original file: Training_1/0_ParallelFrame/0_bisector_1.tex
%\begin{center} \begin{tikzpicture}[scale=0.5]
%    \draw[semithick] (0,0) coordinate (O) circle(\R);
%    \draw (O) node[above] {$O$} circle(1pt);
%    \foreach \i / \name / \pos in {\a/A/110, \b/B/210, \c/C/-30, \e/E/270}{
%        \coordinate[label=\pos:$\name$] (\name) at (\i:\R);
%    }
%    \draw[very thick] (A) -- (B) -- (C) --cycle;
%    \draw[name path=A--E] (A)--(E) ;
%    \path[name path=B--C] (B)--(C) ;
%    \path[name intersections={of=A--E and B--C, by=D}];
%    \draw (D) node[below left] {$D$} circle(1pt);
%    \draw ($(A)!(D)!(B)$) coordinate (G) node[left] {$G$} circle(1pt);
%    \draw ($(A)!(D)!(C)$) coordinate (H) node[left] {$H$} circle(1pt);
%
%    \draw[name path=G--P] (G)--($(G)!2.5!(D)$);
%    \path[name path=A--P] (C) -- ($(A)!3!(C)$);
%    \path[name intersections={of=G--P and A--P, by=P}];
%    \draw (C)--(P) (G)--(P);
%
%    \path[name path=H--Q] (H)--($(H)!6!(D)$);
%    \path[name path=A--Q] (B) -- ($(A)!3!(B)$);
%    \path[name intersections={of=H--Q and A--Q, by=Q}];
%    \draw (B)--(Q) (H)--(Q);
%
%    \draw[name path=A--X] (A) -- ($(A)!1.7!-90:(E)$);
%    \draw[name path=E--Y] (E) -- ($(E)!1!-90:(A)$);
%    \path[name path=X--Y] ($(C)!2.5!(B)$) -- ($(B)!1.5!(C)$);
%
%    \path[name intersections={of=A--X and X--Y, by = X}];
%    \draw (X) node[below] {$X$} -- (B);
%
%    \draw let \p1=($(X)!0.5!(Q)$), \p2=($(X)!0.5!(P)$), 
%        \p3=($(\p1)!1!90:(Q)$), \p4=($(\p2)!1!-90:(P)$),	
%        \p5=(intersection of \p1--\p3 and \p2--\p4)    
%        in coordinate (O_1) at (\p5) ;
%
%    \node[name path=C_1, circle through=(X), draw=blue] at (O_1) {};
%
%\end{tikzpicture} \end{center}
    %\tkzInit[xmax=5,ymax=5]
    %\tkzDrawX[>=latex]
    %\tkzDrawY[>=latex]


\numbering 평행사변형 $ABCD$ 의 한 변 $CD$ 위의 임의의 한 점을 $F$ 라 하자. 
\[ \overline{AE}^2 = \overline{EF} \cdot \overline{EG} \] 
을 보여라. \\
\tikzmath {   
    \R = 3; \myRatio = 0.3; \opac = 0.7;
    \a=115; \b=-155; \c=-25; \e=-65;
}

\colorlet{input}{red!80!black}
\colorlet{output}{red!70!black}
\colorlet{triangle}{green!50!black!40}

\begin{center}
\begin{tikzpicture}[scale=1.2,thick,help lines/.style={thin,draw=black!50}]

    \tkzDefPoint(0,0){A}        \tkzLabelPoint[below left](A){$A$}       
    \tkzDefPoint(3,0){B}        \tkzLabelPoint[below right](B){$ B$}  
    \tkzDefPoint(4,2.5){C}      \tkzLabelPoint[above right](C){$ C$}  
    \tkzDefPoint(1,2.5){D}      \tkzLabelPoint[above left](D){$ D$}  

    \tkzDefBarycentricPoint(C=2,D=1)   \tkzGetPoint{F} \tkzLabelPoint[above](F){$ F$} 
    \tkzInterLL(B,D)(A,F)       \tkzGetPoint{E}     
    \tkzLabelPoint[below, yshift=-0.1cm](E){$ E$}

    \tkzInterLL(B,C)(A,F)       \tkzGetPoint{G}
    \tkzLabelPoint[right, yshift=-0.1cm](G){$ G$}
    
    \tkzDrawPoints(A,B,C,D,E,F,G)
    \tkzDrawSegments[thick](A,G A,B B,D B,G C,D D,A)
\end{tikzpicture}
\end{center}
