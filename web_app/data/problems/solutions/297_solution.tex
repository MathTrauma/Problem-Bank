$\mathrm{AD} \parallel \mathrm{BC}$이므로 $\angle\mathrm{FEG} = \angle\mathrm{CBE}$ (동위각)

사각형 $\mathrm{ABCD}$가 평행사변형이고

$\angle\mathrm{EBA} = \angle\mathrm{CBE}$이므로

\[
\angle\mathrm{FDC} = \angle\mathrm{CBA} = 2 \times \angle\mathrm{CBE}
\]

$\angle\mathrm{FDC} = 2 \times \angle\mathrm{DCF}$이므로 $\angle\mathrm{CBE} = \angle\mathrm{DCF}$

$\angle\mathrm{FEG} = \angle\mathrm{DCF}$

$\angle\mathrm{GFE} = \angle\mathrm{DFC}$ (맞꼭지각)

그러므로 두 삼각형 $\mathrm{GEF}$, $\mathrm{DCF}$에서

$\angle\mathrm{FEG} = \angle\mathrm{FCD}$, $\angle\mathrm{GFE} = \angle\mathrm{DFC}$

삼각형의 세 내각의 크기의 합은 $180^{\circ}$이므로

$\angle\mathrm{EGF} = \angle\mathrm{CDF}$

$\angle\mathrm{GFE} = \angle\mathrm{DFC}$, $\angle\mathrm{EGF} = \angle\mathrm{CDF}$, $\mathrm{FG} = \mathrm{FD} = 4$이므로

삼각형 $\mathrm{GEF}$와 삼각형 $\mathrm{DCF}$는 서로 합동이다.

따라서 $\mathrm{EG} = \mathrm{CD}$이고 사각형 $\mathrm{ABCD}$는 평행사변형이므로 $\mathrm{AB} = \mathrm{CD}$

$\mathrm{AD} \parallel \mathrm{BC}$이므로 $\angle\mathrm{AEB} = \angle\mathrm{CBE}$ (엇각)

$\angle\mathrm{EBA} = \angle\mathrm{CBE}$이므로 $\angle\mathrm{EBA} = \angle\mathrm{AEB}$

삼각형 $\mathrm{ABE}$는 $\mathrm{AE} = \mathrm{AB}$인 이등변삼각형이므로

$\mathrm{AE} = \mathrm{AB}$

선분 $\mathrm{EF}$의 길이를 $x$라 하면

$\mathrm{BC} = \mathrm{AD} = \mathrm{AE} + \mathrm{EF} + \mathrm{FD} = 9 + x + 4$

또한 $\mathrm{EF} = \mathrm{FC} = x$이므로 $\mathrm{GC} = \mathrm{GF} + \mathrm{FC} = 4 + x$

삼각형 $\mathrm{GBC}$에서 $\mathrm{EF} \parallel \mathrm{BC}$이므로

\[
\mathrm{EF} : \mathrm{BC} = \mathrm{GF} : \mathrm{GC}
\]

\[
x : (13+x) = 4 : (4+x)
\]

\[
x(4+x) = 4(13+x)
\]

\[
4x + x^2 = 52 + 4x
\]

\[
x^2 = 52
\]

\[
x = \sqrt{52} = 2\sqrt{13}
\]

(단, $x > 0$)

따라서 선분 $\mathrm{EF}$의 길이는 $2\sqrt{13}$
