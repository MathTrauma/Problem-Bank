% Solution for Problem 001

답 73 \\


\begin{tikzpicture}[scale=0.8]

% 점 정의
\tkzDefPoint(0,0){O}
\tkzDefPoint(-8.54,0){A}
\tkzDefPoint(8.54,0){B}
\tkzDefPoint(1,0){E}
\tkzDefPoint(1,6){M}          % CD-원의 중심
\tkzDefPoint(-4.92,6.99){C}
\tkzDefPoint(6.92,5.01){D}

% 반원 (AB가 지름)
\tkzDrawSemiCircle[thick](O,B)

% 지름 AB
\tkzDrawSegment[thick](A,B)

% CD를 지름으로 하는 원
\tkzDrawCircle[thick, color=blue!60](M,C)

% 선분 CD
\tkzDrawSegment[thick, color=red!70](C,D)

% 접점 E에서 수직선 (접선의 법선)
\tkzDrawSegment[dashed, color=gray](E,M)

% 선분 OE 표시
\tkzDrawSegment[thick, color=green!50!black](O,E)

% 점 표시
\tkzDrawPoints[size=4](A,B,O,E,M,C,D)

% 라벨
\tkzLabelPoint[below left](A){$A$}
\tkzLabelPoint[below right](B){$B$}
\tkzLabelPoint[below](O){$O$}
\tkzLabelPoint[below](E){$E$}
\tkzLabelPoint[right](M){$M$}
\tkzLabelPoint[above left](C){$C$}
\tkzLabelPoint[above right](D){$D$}

% 길이 표시
\tkzLabelSegment[below, pos=0.5](O,E){1}
\tkzLabelSegment[above, pos=0.5, color=red!70](C,D){12}

% 직각 표시 (접점에서)
\tkzMarkRightAngle[size=0.4](M,E,B)

\end{tikzpicture}


접점에서 접선에 수직인 직선은 원의 중심을 지난다.\\
즉, $E$ 에서 $AB$ 에 수직인 직선은 $CD$ 의 중점(원의 중심, 현의 중점)을 지난다. \\
$\overline{CD}$ 의 중점을 $M$ 이라 하자. $\overline{OM} = \sqrt{6^2 + 1}$ 을 얻는다.\\
선분 $OM$ 은 원 $O$ 의 현 $CD$ 에 수직이므로 $\overline{OC} = \sqrt{37 + 6^2}$ 을 얻는다. \\