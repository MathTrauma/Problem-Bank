점 $\mathrm{O}$가 삼각형 $\mathrm{ABC}$의 외심이므로

$\mathrm{OA} = \mathrm{OB} = \mathrm{OC}$

세 삼각형 $\mathrm{OAB}$, $\mathrm{OBC}$, $\mathrm{OCA}$는 이등변삼각형이므로

$\angle\mathrm{OBA} = \angle\mathrm{BAO}$, $\angle\mathrm{CBO} = \angle\mathrm{OCB}$, $\angle\mathrm{ACO} = \angle\mathrm{OAC}$

삼각형 $\mathrm{ABC}$는 $\mathrm{AC} = \mathrm{BC}$인 이등변삼각형이므로

각 $\mathrm{ACB}$의 이등분선은 밑변 $\mathrm{AB}$를 수직이등분한다.

삼각형 $\mathrm{ABC}$의 외심 $\mathrm{O}$는 세 변의 수직이등분선의 교점이므로 점 $\mathrm{O}$는 각 $\mathrm{ACB}$의 이등분선 위에 있다.

$\angle\mathrm{ACO} = \angle\mathrm{OCB}$

삼각형 $\mathrm{ABC}$의 세 내각의 크기의 합은

\begin{align*}
\angle\mathrm{BAC} + \angle\mathrm{CBA} + \angle\mathrm{ACB} &= (\angle\mathrm{BAO} + \angle\mathrm{OAC}) + (\angle\mathrm{CBO} + \angle\mathrm{OBA}) \\
&\quad + (\angle\mathrm{ACO} + \angle\mathrm{OCB}) \\
&= 2 \times \angle\mathrm{BAO} + 2 \times \angle\mathrm{OAC} + 2 \times 2 \times \angle\mathrm{OAC} \\
&= 180^{\circ}
\end{align*}

$\angle\mathrm{OAC} = \dfrac{180^{\circ} - 2 \times 28^{\circ}}{6} = \dfrac{124^{\circ}}{6} = \dfrac{62^{\circ}}{3}$

삼각형 $\mathrm{ACD}$에서 각 $\mathrm{C}$의 외각 $\angle\mathrm{ACB}$의 크기는

$\angle\mathrm{ACB} = \angle\mathrm{CAD} + \angle\mathrm{ADC}$

$\angle\mathrm{CAD} = \angle\mathrm{ACB} - \angle\mathrm{ADC} = 2 \times \dfrac{62^{\circ}}{3} - 40^{\circ} = \dfrac{4^{\circ}}{3}$

삼각형의 내심은 세 내각의 이등분선의 교점이므로

$\angle\mathrm{CAI} = \angle\mathrm{IAD}$

$\angle\mathrm{CAD} = 2 \times \angle\mathrm{CAI}$

$\angle\mathrm{CAI} = \dfrac{1}{2} \times \angle\mathrm{CAD} = \dfrac{1}{2} \times \dfrac{4^{\circ}}{3} = \dfrac{2^{\circ}}{3}$

$\angle\mathrm{OAI} = \angle\mathrm{OAC} - \angle\mathrm{CAI} = \dfrac{62^{\circ}}{3} - \dfrac{2^{\circ}}{3} = \dfrac{60^{\circ}}{3} = 20^{\circ}$

따라서 $x = 20$
