% Solution for Problem 029
% Source: 36회(2022) KMO 중등부 1차 17번
% Original file: Problem Sets/contents/quiz05.tex


\fbox{ 36회(2022) KMO 중등부 1차 17번 }\\ 
36회는 난이도가 순서대로 높아지지 않았다. 이 문제는 5점\\
답 : 17 \\

\begin{tikzpicture}[scale=.6]
    % 외접원 중심과 반지름
    \tkzDefPoint(0,0){O}
    \tkzDefPoint(3,0){P}

    % 삼각형 ABC 정의 (외접원 위의 점들)
    \tkzDefPoint(50:3){A}
    \tkzDefPoint(190:3){B}
    \tkzDefPoint(-30:3){C}

    % 외접원 그리기
    \tkzDrawCircle[color=blue](O,P)

    % 점 A에서의 접선 (OA에 수직)
    \tkzDefLine[tangent at=A](O) \tkzGetPoint{t}

    % 직선 BC와 접선의 교점 D
    \tkzInterLL(A,t)(B,C) \tkzGetPoint{D}

    % M은 AD의 중점
    \tkzDefMidPoint(A,D) \tkzGetPoint{M}

    % BM과 외접원의 교점 E (B가 아닌 점 선택)
    \tkzInterLC(B,M)(O,P) \tkzGetPoints{E1}{E2}
    \tkzGetPointCoord(B){Bp}
    \tkzGetPointCoord(E1){Ea}
    \tkzGetPointCoord(E2){Eb}
    % E1과 B의 거리가 0.1보다 작으면 E2 선택, 아니면 E1 선택
    \pgfmathsetmacro{\distE}{sqrt((\Eax-\Bpx)^2+(\Eay-\Bpy)^2)}
    \ifdim \distE pt<0.1pt
        \tkzDefPoint(\Ebx,\Eby){E}
    \else
        \tkzDefPoint(\Eax,\Eay){E}
    \fi

    % BED의 외접원
    \tkzDefCircle[circum](B,E,D) \tkzGetPoint{Obed}

    % 삼각형과 선분 그리기
    \tkzDrawPolygon(A,B,C)
    \tkzDrawSegments(A,D B,E E,A E,C E,M E,D)
    \tkzDrawSegment(C,D)
    \tkzDrawLine[dashed,add=0.1 and 0.3](B,C)

    % BED 외접원 (clip)
    \begin{scope}
        % \tkzClipCircle(O,P)
        \tkzDrawCircle[color=red,line width=1.2pt](Obed,B)
    \end{scope}

    % 점 표시
    \tkzDrawPoints(A,B,C,D,E,M)
    \tkzLabelPoint[above right](A){$A$}
    \tkzLabelPoint[left](B){$B$}
    \tkzLabelPoint[below right](C){$C$}
    \tkzLabelPoint[below left](D){$D$}
    \tkzLabelPoint[above left](E){$E$}
    \tkzLabelPoint[above](M){$M$}
\end{tikzpicture}
