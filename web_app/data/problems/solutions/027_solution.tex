% Solution for Problem 027
% Source: 36회(2022) KMO 중등부 1차 8번
% Original file: Problem Sets/contents/quiz04.tex


\fbox{ 36회(2022) KMO 중등부 1차 8번 }\\ 
36회 문제 배열은 난이도 순이 아니었다. 6점\\
답 : 39 \\

\begin{tikzpicture}[scale=0.2]
    % 삼각형 ABC (AB=27, BC=30)
    \tkzDefPoint(0,0){B}
    \tkzDefPoint(30,0){C}
    % A는 B를 중심으로 반지름 27인 원과 C를 중심으로 반지름 39인 원의 교점
    \tkzDefCircle[R](B,27) \tkzGetPoint{b}
    \tkzDefCircle[R](C,39) \tkzGetPoint{c}
    \tkzInterCC(B,b)(C,c) \tkzGetPoints{A}{A'}

    % M은 BC의 중점
    \tkzDefMidPoint(B,C) \tkzGetPoint{M}

    % BC를 지름으로 하는 원
    \tkzDrawCircle[color=blue](M,B)

    % AM과 원의 교점 D (y좌표가 양수인 점 선택)
    \tkzInterLC(A,M)(M,B) \tkzGetPoints{D1}{D2}
    \tkzGetPointCoord(D1){Da}
    \tkzGetPointCoord(D2){Db}
    \ifdim \Day pt>0pt
        \tkzDefPoint(\Dax,\Day){D}
    \else
        \tkzDefPoint(\Dbx,\Dby){D}
    \fi

    % E: 직선 CD와 AB의 교점
    \tkzInterLL(C,D)(A,B) \tkzGetPoint{E}

    % F: 직선 BD와 AC의 교점
    \tkzInterLL(B,D)(A,C) \tkzGetPoint{F}

    % 삼각형과 선분 그리기
    \tkzDrawPolygon(A,B,C)
    \tkzDrawSegments(A,M C,E B,F E,F)

    % 점 표시
    \tkzDrawPoints(A,B,C,D,E,F,M)
    \tkzLabelPoint[above](A){$A$}
    \tkzLabelPoint[below left](B){$B$}
    \tkzLabelPoint[below right](C){$C$}
    \tkzLabelPoint[below](D){$D$}
    \tkzLabelPoint[left](E){$E$}
    \tkzLabelPoint[right](F){$F$}
    \tkzLabelPoint[below](M){$M$}
\end{tikzpicture}
