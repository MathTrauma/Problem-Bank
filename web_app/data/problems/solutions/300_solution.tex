삼각형 $\mathrm{ABC}$의 넓이가 $204$이므로

\[
\triangle\mathrm{ABC} = \dfrac{1}{2} \times \mathrm{BC} \times \mathrm{AH} = \dfrac{1}{2} \times 17 \times \mathrm{AH} = 204
\]

$\mathrm{AH} = 24$

직각삼각형 $\mathrm{AHC}$에서 피타고라스 정리에 의하여

\[
\mathrm{AC}^2 = \mathrm{AH}^2 + \mathrm{HC}^2
\]

\[
\mathrm{HC}^2 = \mathrm{AC}^2 - \mathrm{AH}^2 = 26^2 - 24^2 = 100
\]

$\mathrm{HC} = 10$

$\mathrm{BH} = \mathrm{BC} - \mathrm{HC} = 17 - 10 = 7$

직각삼각형 $\mathrm{ABH}$에서 피타고라스 정리에 의하여

\[
\mathrm{AB}^2 = \mathrm{BH}^2 + \mathrm{AH}^2 = 7^2 + 24^2 = 625
\]

$\mathrm{AB} = 25$

점 $\mathrm{I}$에서 세 변 $\mathrm{AB}$, $\mathrm{BC}$, $\mathrm{CA}$에 내린 수선의 발을 각각 $\mathrm{J}$, $\mathrm{K}$, $\mathrm{L}$이라 하자.

삼각형 $\mathrm{ABC}$의 내접원의 반지름의 길이를 $r$이라 하면

삼각형의 내심의 성질에 의하여

$\mathrm{IJ} = \mathrm{IK} = \mathrm{IL} = r$

\[
\triangle\mathrm{ABC} = \triangle\mathrm{IAB} + \triangle\mathrm{IBC} + \triangle\mathrm{ICA} = \dfrac{1}{2} \times \mathrm{AB} \times r + \dfrac{1}{2} \times \mathrm{BC} \times r + \dfrac{1}{2} \times \mathrm{CA} \times r
\]

\[
= \dfrac{1}{2} \times r \times (25 + 17 + 26) = 34r
\]

$204 = 34r$에서 $r = 6$

두 직각삼각형 $\mathrm{IKC}$, $\mathrm{ILC}$에서

$\angle\mathrm{IKC} = \angle\mathrm{ILC} = 90^{\circ}$, $\mathrm{IK} = \mathrm{IL}$, $\mathrm{IC}$는 공통이므로

두 직각삼각형 $\mathrm{IKC}$, $\mathrm{ILC}$는 서로 합동이다.

그러므로 $\mathrm{CK} = \mathrm{CL}$

같은 방법으로

두 직각삼각형 $\mathrm{ILA}$, $\mathrm{IJA}$가 서로 합동이고 $\mathrm{AL} = \mathrm{AJ}$,

두 직각삼각형 $\mathrm{IJB}$, $\mathrm{IKB}$가 서로 합동이고 $\mathrm{BJ} = \mathrm{BK}$

$\mathrm{HK} = a$라 하면

$\mathrm{CK} = \mathrm{CL} = 10 - a$

$\mathrm{AL} = \mathrm{AJ} = 24 - (10-a) = 14 + a$

$\mathrm{BJ} = \mathrm{BK} = 7 - a$

\[
\mathrm{BC} = \mathrm{BK} + \mathrm{KC} = (7-a) + (10-a) = 17 - 2a = 17
\]

$-2a = 0$, $a = 0$

직각삼각형 $\mathrm{IHK}$에서 $\mathrm{HK} = 0$, $\mathrm{IK} = 6$

피타고라스 정리에 의하여

\[
\mathrm{IH}^2 = \mathrm{HK}^2 + \mathrm{IK}^2 = 0^2 + 6^2 = 36
\]
