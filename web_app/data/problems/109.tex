% Problem ID: 109
% Original file: Training_1/0_ParallelFrame/0_bisector_2.tex

\numbering 다음 그림의 $ABCD$는 한변의 길이가 1인 정사각형이다. \\
(1) 그림의 어떤 선분의 길이가 $x$로 주어졌을 때, 
다른 선분들의 길이를 $x$로 표현하여라. \\

\begin{tikzpicture}[scale=0.8]
    \def\t{3}
    \tkzDefPoints{0/0/A, 6/0/B, 6/6/C, 0/6/D}
    \tkzLabelPoint[below left](A){$A$} 
    \tkzLabelPoint[below right](B){$B$} 
    \tkzLabelPoint[above right](C){$C$} 
    \tkzLabelPoint[above left](D){$D$} 
    
    \tkzDefPoint(\t,6){A1}   \tkzLabelPoint[above](A1) {$A'$}
    \tkzDrawPoints(A,A1,B,C,D)
    
    %\tkzDefPointOnLine[pos=.6](A,B) \tkzGetPoint{E}
    %\tkzDrawPoint(E)
    
    \tkzDefLine[mediator](A,A1) \tkzGetPoints{X}{Y}
    \tkzInterLL(A,D)(X,Y)  \tkzGetPoint{P}
    \tkzInterLL(B,C)(X,Y)  \tkzGetPoint{Q}
    \tkzDrawPoints(P,Q) \tkzDrawSegment(P,Q)
    \tkzLabelPoint[left](P){$P$}    \tkzLabelPoint[below right](Q){$Q$} 
    
    \tkzDefPointBy[reflection= over P--Q](B)    \tkzGetPoint{B1}
    \tkzDrawPoint(B1)   \tkzLabelPoint[above right](B1){$B'$}
    
    \tkzDrawPolygon[thick, black, fill=blue!30, opacity=.5](P,A1,B1,Q)

    \tkzInterLL(A1,B1)(B,C)     \tkzGetPoint{R}
    \tkzDrawPoint(R)        \tkzLabelPoint[above right](R){$R$}
    
    \tkzDrawSegments[thick](P,D D,C C,R)
    \tkzDrawSegments[thick, densely dashed](A,P A,B B,Q)
    \tkzDrawSegments[semithick, dashed](Q,R)
\end{tikzpicture}