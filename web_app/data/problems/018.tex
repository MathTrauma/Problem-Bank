% Problem ID: 018
% Original file: Problem Sets/contents/99_Easy.tex

\numbering 삼각형 $ABC$에서 선분 $BC$의 중점을 $M$, 선분 $AM$의 중점을 $N$이라 하자.
직선 $BN$과 선분 $AC$의 교점을 $D$라 할 때, $\overline{AD} : \overline{CD}$를 구하시오.\\ 

\begin{center}
    \begin{tikzpicture}[scale=1.2,semithick,help lines/.style={thin,draw=black!50}]
        %\tkzInit[xmax=5,ymax=5]
        %\tkzDrawX[>=latex]
        %\tkzDrawY[>=latex]
    
        \tkzDefPoint(-2,0){B}        \tkzLabelPoint[below](B){$B$}       
        \tkzDefPoint(2,0){C}        \tkzLabelPoint[below](C){$C$}  
        \tkzDefPoint(1,3){A}      \tkzLabelPoint[above](A){$A$}  
        \tkzDefPoint(0,0){M}      \tkzLabelPoint[below](M){$M$}  
    
        \tkzDefBarycentricPoint(A=1,M=1)   \tkzGetPoint{N} 
        \tkzLabelPoint[below right](N){$N$} 

        \tkzInterLL(B,N)(A,C)       \tkzGetPoint{D}     
        \tkzLabelPoint[right, xshift=0.1cm](D){$D$}

        \tkzDrawSegments[semithick](A,B B,C C,A B,D A,M)

    \end{tikzpicture}
\end{center}

두 가지 예시.
\begin{figure}[H]
	\begin{subfigure}{0.5\textwidth}
		\centering
        \begin{tikzpicture}[scale=1.2,semithick,help lines/.style={thin,draw=black!50}]
            \tkzDefPoint(-2,0){B}        \tkzLabelPoint[below](B){$B$}       
            \tkzDefPoint(2,0){C}        \tkzLabelPoint[below](C){$C$}  
            \tkzDefPoint(1,3){A}      \tkzLabelPoint[above](A){$A$}  
            \tkzDefPoint(0,0){M}      \tkzLabelPoint[below](M){$M$}  
        
            \tkzDefBarycentricPoint(A=1,M=1)   \tkzGetPoint{N} 
            \tkzLabelPoint[below right](N){$N$} 

            \tkzInterLL(B,N)(A,C)       \tkzGetPoint{D}     
            \tkzLabelPoint[right, xshift=0.1cm](D){$D$}

            \tkzDefLine[parallel=through A](B,C)    \tkzGetPoint{X}
            \tkzInterLL(A,X)(B,N)   \tkzGetPoint{E}
            \tkzLabelPoint[above](E){$E$}
    
            \tkzDrawSegments[semithick](A,B C,A B,D A,M D,E B,C A,E)
            \tkzMarkSegments[mark=arrow, size=3pt](B,C A,E)
    
        \end{tikzpicture}
        \caption{$\frac{\overline{AE}}{\overline{EF}} = \frac{\overline{BE}}{\overline{ED}}$}
	\end{subfigure}  %%%%%%%%%
	\begin{subfigure}{0.5\textwidth}
		\centering
        \begin{tikzpicture}[scale=1.2,semithick,help lines/.style={thin,draw=black!50}]
            \tkzDefPoint(-2,0){B}        \tkzLabelPoint[below](B){$B$}       
            \tkzDefPoint(2,0){C}        \tkzLabelPoint[below](C){$C$}  
            \tkzDefPoint(1,3){A}      \tkzLabelPoint[above](A){$A$}  
            \tkzDefPoint(0,0){M}      \tkzLabelPoint[below](M){$M$}  
        
            \tkzDefBarycentricPoint(A=1,M=1)   \tkzGetPoint{N} 
            \tkzLabelPoint[below right](N){$N$} 

            \tkzInterLL(B,N)(A,C)       \tkzGetPoint{D}     
            \tkzLabelPoint[right, xshift=0.1cm](D){$D$}

            \tkzDefLine[parallel=through M](B,N)    \tkzGetPoint{X}
            \tkzInterLL(M,X)(A,C)       \tkzGetPoint{E}
            \tkzLabelPoint[right](E){$E$}

            \tkzDrawSegments[semithick](A,B B,C A,C A,M B,D M,E)
            
        \end{tikzpicture}
        \caption{$\frac{\overline{BE}}{\overline{ED}} = \frac{\overline{EG}}{\overline{AE}}$}
	\end{subfigure}
	%%%%%%%%%%%%%%%%%%%%%%%%%%%%%%%%%
	%\caption{Caption}
\end{figure}