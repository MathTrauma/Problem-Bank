그림과 같이 평행사변형 $\mathrm{ABCD}$에서 각 $\mathrm{B}$의 이등분선이 선분 $\mathrm{AD}$와 만나는 점을 $\mathrm{E}$라 하자. 선분 $\mathrm{ED}$ 위의 한 점 $\mathrm{F}$에 대하여 $\angle\mathrm{FDC}=2\times\angle\mathrm{DCF}$이다. 직선 $\mathrm{BE}$와 직선 $\mathrm{CF}$의 교점을 $\mathrm{G}$라 할 때, $\mathrm{EG}=3$, $\mathrm{FG}=\mathrm{FD}=2$이다. 선분 $\mathrm{EF}$의 길이는?

% [SVG: 297_fig1.png]

\begin{enumerate}[label=\circled{\arabic*}, itemsep=0pt]
\item $3$
\item $\sqrt{10}$
\item $\sqrt{11}$
\item $2\sqrt{3}$
\item $\sqrt{13}$
\end{enumerate}
